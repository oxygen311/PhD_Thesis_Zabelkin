\chapter{Байесовская оптимизация для вывода параметров заданной модели демографической истории популяций}
\label{sec:part4}

В данной главе представлены описания и результаты экспериментальных исследований методов вывода демографической истории популяций, разработанных в данной работе.
Для этого был использован реализованный программный комплекс GADMA.

Раздел~\ref{sec:part4:genetic_algorithm} включает в себя экспериментальные исследования по выявлению эффективности метода генетического алгоритма, а также метода автоматического построения модели демографической истории популяций.
Для этого использовались как симулированные данные, так и ранее опубликованный эмпирические данные.
Демонстрируется, что разработанный метод генетического алгоритма способен найти значения параметров модели демографической истории, которые имеют лучшее значение правдоподобия, чем параметры, полученные существующими до этого алгоритмами локального поиска.
Также показано, что метод автоматического поиска модели способен отобразить изначальную историю, а также подобрать реалистичные динамики изменения численности популяций автоматически.

В разделе~\ref{sec:part4:hpo} представлены результаты экспериментов по поиску гиперпараметров разработанного генетического алгоритма с использованием метода, реализованного в программном обеспечении SMAC.
В результате, была получена более эффективная конфигурация генетического алгоритма.
Также в разделе представлены результаты применения обновленного генетического алгоритма для сравнения различных методов вычисления правдоподобия для задачи вывода демографической истории, таких как \dadi, \moments, \momi и \momentsLD, на симулированных данных, а также выведенные демографические истории с коэффициентами инбридинга для двух эмпирических датасетов.

Результаты экспериментальных исследований сравнения разных кандидатов 
метода байесовской оптимизации для решения задачи вывода демографической истории представлены в разделе~\ref{sec:part4:bayesian_optimization}.
Наилучший ансамблевый вариант метода сравнивается с генетическим алгоритмом на симулированных и реальных данных одной, двух, трех, четырех и пяти популяций.
Кроме этого, метод ансамбля байесовской оптимизации применяется для вывода демографической истории четырех и пяти популяций современного человека.

Последний раздел~\ref{sec:part4:application} данной главы включает в себя применение разработанных методов для поиска демографической истории популяций по генетическим данным, которые до этого не были проанализированы.
GADMA была применена для данных трех популяций современных людей, а также для трех популяций голубых акул.



