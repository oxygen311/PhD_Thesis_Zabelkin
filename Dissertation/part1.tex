\chapter{Обзор предметной области}
\label{chap:overview}

\section{Модели случайных графов}
\label{sec:random_graph_models}

\subsection{Классическая модель Эрдеша–Реньи}
\label{subsec:erdos_renyi}

\subsection{Порог появления гигантской компоненты}
\label{subsec:giant_component_threshold}

\subsection{Аффинные модификации модели случайных графов}
\label{subsec:affine_modifications}

\subsection{Граф точек разрыва и его аффинные модификации}
\label{subsec:breakpoint_graph}

\section{Постановка задачи оценки расстояний между структурами}
\label{sec:distance_estimation}

\subsection{Метрика минимального числа операций}
\label{sec:minimal_operations}

\subsection{Вероятностная модель поломки случайных регионов}
\label{subsec:random_breakage}

\subsection{Вероятностная модель поломки хрупких регионов}
\label{subsec:fragile_breakage}

\section{Методы анализа параллельных изменений в древовидных структурах}
\label{sec:dirichlet_model}

\subsection{Снабжение графа точек разрыва весами}
\label{subsec:breakpoint_weights}

\subsection{Выпуклые признаки на деревьях}
\label{subsec:modified_dcj}

\subsection{Литературные примеры параллельных изменений}
\label{subsec:parallel_changes_examples}

\subsection{Задача оценки степени параллельности}
\label{subsec:parallelism_estimation}
