\chapter{Введение и обзор существующих подходов}
\label{chap:introduction}

\section{Классические модели случайных графов}
\label{sec:random_graph_models}

\subsection{Модель случайного графа Эрдёша–Реньи}
\label{subsec:erdos_renyi}

\subsection{Модели графов с неоднородными вероятностями}
\label{subsec:heterogeneous_graphs}

\section{Методы оценки эволюционного расстояния}
\label{sec:evolutionary_distance_methods}

\subsection{Оценка через минимальное расстояние}
\label{subsec:minimal_distance}

\subsection{Модель поломки случайных регионов}
\label{subsec:random_breakage}

\subsection{Модель поломки хрупких регионов}
\label{subsec:fragile_breakage}

\section{Анализ модели поломки хрупких регионов}
\label{sec:fragile_model_analysis}

\subsection{Граф точек разрыва и двойной-разрез-и-склеивание}
\label{subsec:breakpoint_graph_dcj}

\subsection{Эволюционная модель и оценки расстояний}
\label{subsec:evolutionary_model_distance}

\section{Модель случайных графов с Дирихле-распределёнными вероятностями}
\label{sec:dirichlet_model}

\subsection{Снабжение графа точек разрыва весами}
\label{subsec:breakpoint_weights}

\subsection{Модифицированная операция двойной-разрез-и-склеивание}
\label{subsec:modified_dcj}

\subsection{Равновесное распределение и эволюционная модель}
\label{subsec:dirichlet_equilibrium}
