%!TEX root = ../dissertation.tex
\begin{itemize}
    \item проведено исследование текущего состояния предметной области, уточнение задачи и способов оценки результатов;
    \item формализована постановка задачи построения и настройки моделей метрических деревьев с функциями на ребрах на примере задачи вывода демографической истории популяций по генетическим данным;
    \item разработан метод автоматической настройки параметров моделей метрических деревьев с функциями на ребрах на основе комбинации методов глобальной и локальной оптимизации на примере задачи вывода демографической истории популяций по генетическим данным;
    \item разработан метод автоматического перебора моделей метрических деревьев с функциями на ребрах на примере задачи вывода демографической истории популяций по генетическим данным;
    \item спроектирован и реализован программный комплекс, включающий разработанные модели и методы для вывода демографической истории популяций по генетическим данным;
    \item проведены экспериментальные исследования, подтверждающие эффективность разработанных моделей и методов, а также их применимость для вывода демографической истории популяций по генетическим данным, проведен анализ результатов экспериментов.\\
\end{itemize}

Для оценки качества настройки моделей демографических историй в данной работе было использовано значение функции правдоподобия.
Результаты экспериментов показывают, что метод настройки параметров моделей на основе комбинации генетического алгоритма и локального поиска позволил в 88\% случаев (37 моделей из 42 протестированных) найти параметры модели, обеспечивающие лучшее значение правдоподобия, чем параметры, найденные существующими ранее методами.
На симулированных данных разработанный метод позволил найти решения, которые на 97\% ближе к оптимуму в случае одной популяции и на 66\% ближе к оптимуму в случае трех популяций, чем решения, полученные существующими методами.
Настройка гиперпараметров генетического алгоритма позволила ускорить реализацию в среднем на 10\% с сохранением эффективности метода.

Была подтверждена эффективность метода настройки параметров моделей на основе байесовской оптимизации и локальной оптимизации в условиях сложновычислимной целевой функции.
Разработанный метод позволил найти значения параметров, обеспечивающих лучшее значение правдоподобия, чем существующие методы, для двух ранее проанализированных данных четырех и пяти популяций.
Было показано, что байесовская оптимизация достигает решения, близкого к оптимуму, на 50-80\% быстрее, чем генетический алгоритм, в случае вывода демографической истории четырех и пяти популяций.

Метод автоматического перебора моделей позволяет автоматически строить и настраивать модели в заданных ограничениях на конфигурацию.
Сравнение моделей демографических историй с разным числом параметров было осуществлено с использованием информационного критерия Акаике (AIC).
Экспериментальные исследования показали, что в трех из четырех случаях метод позволил найти модель, обеспечивающую лучшее значение AIC, чем было получено ранее ручным перебором.
В четвертом случае, полученная модель позволила установить излишние параметры в конфигурации и построить вложенную модель, которая в итоге обеспечила наилучшее значение AIC для данных.

В качестве перспективных направлений исследования можно выделить совершенствование метода автоматического перебора моделей с целью поиска оптимального набора параметров конфигурации, а также разработку методов настройки моделей метрического дерева с функциями на ребрах, которые позволяют осуществлять настройку не только функциональных параметров, но и поиск оптимальной структуры дерева.