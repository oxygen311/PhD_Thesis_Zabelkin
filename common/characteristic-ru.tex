%!TEX root = ../dissertation.tex

\nociteallauthorpublicationsmaindocument

\textbf{Тема исследования.}
Методы моделирования нестационарных стохастических процессов, основанные на представлении объектов реального мира в виде графов и алгоритмы комбинаторного анализа их паттернов. 
Методы моделирования дискретных случайных процессов на основе комбинаторного анализа перестановок и представлении объектов реального мира в виде графов.

\textbf{Актуальность темы исследования.}
Задача количественной оценки различий между дискретными структурами (например, последовательностями или перестановками) и их эволюции актуальна для прикладной математики и информатики. Она сводится к вычислению расстояния между двумя конфигурациями. Подобные метрики применяются в разных областях: сравнение сетевых графов и маршрутов, анализ текстовых данных, прогнозирование во временных рядах. В частности, широко изучена задача сортировки перестановки за минимальное число операций (например, разворотов последовательности), для которой разработаны как точные алгоритмы, так и статистические оценки.

Для эффективного решения подобных задач требуется строгая математическая формализация. В настоящей работе рассмотрены два аспекта проблемы. Первый – моделирование случайных операций над последовательностями и оценка расстояния между ними. Этот аспект опирается на вероятностные модели (например, марковские процессы), позволяющие рассчитать ожидаемые параметры структур после определённого числа случайных перестановок. Использование таких моделей осложнено тем, что ранее полученные решения опирались на бесконечные рядовые разложения и численные методы оптимизации. Это приводило к вычислительным трудностям: отсутствие сходимости на больших масштабах изменений и высокая сложность реализации алгоритмов (например, необходимость градиентного спуска по системе уравнений). Таким образом, актуальной является задача вывода аналитических формул и разработка более устойчивых методов оценивания расстояния, устраняющих указанные недостатки. Решение этой задачи имеет значение для сценариев сравнения сложных объектов, описываемых перестановками или графами.

Второй аспект – обнаружение параллельных (независимых) изменений в эволюции структуры на заданном древовидном пространстве состояний. Иными словами, когда имеется дерево, вершины которого соответствуют различным версиям структуры (например, различным конфигурациям элементов), важно выявлять те изменения, которые произошли более чем один раз независимо на разных ветвях дерева. Выявление таких конвергентных изменений актуально для общей теории эволюционных процессов и анализа данных на деревьях (иерархиях). Аналогичные задачи встречаются при изучении сходно возникающих черт в языках или культурах, при анализе версий программного кода, и в других областях, где дерево изменений служит моделью развития объекта. С формальной точки зрения, это сводится к задаче проверки консистентности признаков на дереве: если некоторый признак (характеристика конфигурации) не может быть объяснён единственным изменением на данном дереве, значит, имели место независимые (параллельные) события изменения этого признака. Задача автоматической идентификации таких признаков требует строгого определения и алгоритмического решения. Ранее она решалась вручную либо разрозненными методами и не включала методы оценки степени параллельности появления признака. Разработка алгоритмов обнаружения параллельных изменений и метода оценки степени параллельности появления признака может найти применение в любых исследованиях, где нужно обнаруживать сходные паттерны, возникающие независимо, – от сравнительного анализа сетевых структур до обнаружения аномалий в эволюции данных.

\newpage
\textbf{Степень разработки проблемы.}
Расстояние между дискретными структурами – ключевой показатель сходства/различия, и для его вычисления разработаны специальные метрики и алгоритмы. Рассмотрим такие алгоритмы на примере геномных последовательностей. Дискретной структурой можно считать перестановку блоков (генов). Классическая задача – сортировка перестановки минимальным числом операций – лежит в основе оценки эволюционной дистанции между геномами. Ещё в 1990-х были сформулированы методы вычисления расстояния перестановок: так, Bafna и Pevzner ввели понятие циклического графа (cycle graph), позволяющего разложить перестановку на минимальное число операций типа инверсий или транспозиций. Минимальное число таких операций (например, инверсий фрагментов генома) даёт парсимонийную оценку расстояния между геномами. Однако для отдалённых объектов парсимонийная оценка ненадёжна – она систематически занижает истинное расстояние. Современные методы стремятся оценить истинную эволюционную дистанцию, не опираясь только на минимум. Ранее Lin и Moret (2008) также предлагали оценивать «истинное» расстояние под моделью случайных перестроек (DCJ-модель). Alexeev Alekseyev (2017) предложили статистический метод оценки истинного числа геномных перестроек с учётом модели fragile breakage, учитывающую невозможность разрывов в некоторых регионах. В работе Biller et al. (2016) были выведены формулы для оценки расстояния с учётом «хрупких» регионов с разными вероятностями поломки, однако эти формулы представляют бесконечные ряды и требуют численного решения системы уравнений.

Аналогичные метрики разработаны и для других дискретных структур. Например, редакционное расстояние графов (graph edit distance) измеряет различия между двумя графами через минимальную последовательность правок. Этот подход широко применяется для сравнения структурированных данных (включая молекулярные или сетевые структуры в биоинформатике), хотя вычисление точного расстояния редактирования графов является NP-трудной задачей. Для ускорения оценки таких расстояний используются как классические аппроксимации, так и современные методы машинного обучения.

Другим актуальным направлением стало выявление параллельных (конвергентных) изменений на эволюционных деревьях. Если какое-то изменение (например, инверсия фрагмента) неоднократно возникло независимо в разных ветвях дерева, оно считается параллельным. Алгоритмы парсимонии (например, метод Фитча) выявляют минимум изменений, но не дают количественной меры степени параллельности. 

\textbf{Научная новизна диссертации} состоит в том, что: 
\begin{enumerate}[label=(\arabic*)]
    \item Впервые получены аналитические формулы, позволяющие вычислять расстояние между перестановками без суммирования расходящихся рядов и обеспечивающие более высокую точность на больших масштабах перестроек.
    \item Разработан математический метод оценивания эволюционного расстояния между двумя перестановками, допускающими неравномерные вероятности затрагивания отдельных элементов.
    \item Разработан метод автоматического выявления и ранжирования параллельных (независимых) событий на деревьях. Введена новая комбинаторная метрика параллельности, позволяющая количественно оценивать степень независимого многократного возникновения перестановок в разных ветвях дерева.
\end{enumerate}

\textbf{Теоретическая значимость работы} определяется: 
\begin{itemize}
    \item выводом замкнутых аналитических выражений для ключевых статистик циклограммы перестановки, что позволяет более глубоко исследовать марковские процессы на пространстве перестановок;
    \item обобщением алгоритмов парсимонии и методов комбинаторного анализа на случай параллельных изменений, что может быть адаптировано для других классов дискретных моделей и стохастических процессов.
\end{itemize}

\textbf{Практическую значимость работы} определяют:
\begin{itemize}
    \item повышенная точность оценки эволюционных расстояний в задачах сравнительного анализа сложных дискретных структур;
    \item открытая реализация разработанного алгоритма TruEst, который предоставляет инструмент для вычисления расстояния между дискретными структурами;
    \item реализация алгоритма PaReBrick для автоматического обнаружения и ранжирования параллельных перестроек на деревьях, который находит независимые сходные изменения и даёт количественную оценку их значимости.
\end{itemize}

\textbf{Цели и задачи исследования.}
Исследование направлено на разработку математического аппарата для анализа сложных перестановок и их изменений. В рамках работы преследуются две цели:
\begin{enumerate}
    \item Разработка математической модели и методов оценивания расстояния между двумя дискретными последовательностями с учётом неоднородностей в вероятностях отдельных перестановок. Необходимо получить аналитические зависимости, позволяющие вычислять расстояние без численного суммирования расходящихся рядов, и на их основе создать метод эффективного оценивания истинного расстояния между последовательностями.
    \item Метод обнаружения параллельных изменений и оценки степени их параллельности на заданном дереве отношений между объектами. Требуется строго определить критерии независимого возникновения схожих перестановок на разных ветвях дерева и разработать алгоритм, выявляющий и ранжирующий такие случаи (вводя количественную меру "степени параллельности" события).
\end{enumerate}

\textbf{Задачи:}
\begin{itemize}
    \item Вывод аналитических формул для оценки ожидаемого количества ключевых компонент.
    \item Разработка алгоритма оценивания расстояния на основе выведенных формул.
    \item Разработка алгоритма идентификации параллельных перестановок и меры значимости.
\end{itemize}

\textbf{Положения, выносимые на защиту:}
\begin{enumerate}
    \item Метод оценки расстояния между объектами, представленными перестановками на основе марковских процессов с неоднородными вероятностями состояний, отличающийся от существующих подходов тем, что, с целью повышения точности оценки на больших расстояниях, вместо численного суммирования расходящихся рядов используются полученные аналитические выражения для статистических характеристик циклических структур перестановок, что позволило реализовать устойчивое вычисление расстояний даже при высоком уровне изменений состояний объектов.
    
    \item Метод выявления и ранжирования независимых изменений в наборах перестановок на древовидных структурах, основанный на представлении изменений как дискретных признаков и проверке их консистентности с древовидной топологией, отличающийся тем, что, с целью автоматического и объективного выявления повторяющихся событий, предложена новая комбинаторная метрика — «показатель параллельности», количественно отражающая частоту, количество и степень распределённости независимых изменений по вершинам древовидной структуры и позволяющая повысить достоверность анализа.
\end{enumerate}

\textbf{Методы исследования.} В работе использованы методы оптимизации, теории вероятностей, численные и комбинаторные методы, алгоритмы на деревьях, а также вычислительные эксперименты на синтетических и реальных данных на примере данных геномов.

\textbf{Достоверность научных результатов} обеспечена строгими математическими доказательствами, проверкой на симулированных данных, сопоставлением результатов с опубликованными исследованиями и открытостью программного кода, что позволяет независимо подтвердить правильность выводов.

\textbf{Соответствие паспорту специальности.} Полученные научные результаты соответствуют следующим пунктам паспорта специальности 1.2.2 — «Математическое моделирование, численные методы и комплексы программ (технические науки)»:

\begin{itemize}
    \item \textbf{Пункт 2:} «Разработка, обоснование и тестирование эффективных вычислительных методов с применением современных компьютерных технологий». Были разработаны, обоснованы и протестированы методы вычисления расстояний между дискретными структурами и алгоритмы автоматического обнаружения параллельных событий на деревьях, использующие численную оптимизацию и комбинаторный анализ.
    
    \item \textbf{Пункт 4:} «Разработка новых математических методов и алгоритмов интерпретации натурного эксперимента на основе его математической модели». В работе предложены математические методы и алгоритмы интерпретации натурного эксперимента на примере геномных данных, основанные на разработанных вероятностных моделях случайных перестановок.
\end{itemize}