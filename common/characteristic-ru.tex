%!TEX root = ../dissertation.tex

\nociteallauthorpublicationsmaindocument

\textbf{Актуальность темы исследования.}  
В задачах прикладной математики и информатики часто требуется формальное описание и количественная оценка различий между сложными дискретными структурами --- перестановками, графами и деревьями, а также анализ процесса эволюции во времени этих структур~\cite{Penny2004,Bunke1997}.  
Подобные задачи возникают в разных областях: от изучения аудиовизуальной и текстовой информации до моделирования социальных сетей и сопоставления топологий~\cite{baret2004phylogenetic,McCollum2023,Piar2020,Newman2003}.  
Классическим примером является вычисление ``расстояния'' между двумя перестановками (минимального числа операций, переводящих одну конфигурацию в другую), что лежит в основе алгоритмов сортировки перестановок, анализа редактирования графов (англ. \textit{graph edit distance}), а также ряда других задач структурного сравнения~\cite{Pevzner03}.  

Однако, когда речь заходит о процессе изменений, ситуацию усложняет тот факт, что операции над структурой (добавление рёбер в графы, перестановки элементов, модификации вершин дерева) могут происходить случайным образом с неоднородными вероятностями.  
В одних случаях вероятность операции считается одинаковой для всех элементов, как в классической модели \mbox{Эрдёша--Реньи} (равновероятное появление рёбер)~\cite{Erdos1959}, в других же требуется учесть разные ``аффинности'' отдельных элементов.  
Такое неравномерное распределение вероятностей оказывается востребованным при моделировании социальных сетей, процессов обработки и передачи информации, соавторства текстов, взаимодействия порядка генов и т.~п.  

Кроме того, ещё одной важной проблемой является обнаружение параллельных (независимых) изменений в процессах на древовидном пространстве состояний.  
Если в вершинах дерева находятся разные версии исходного объекта (программного кода, текстовой традиции, биологической структуры и т.~д.), то нередко интересуют изменения, которые возникли неоднократно и независимо друг от друга на разных ветвях дерева.  
Подобные конвергентные события важно выявлять в лингвистике (одинаковые языковые новации в независимых группах), в программной инженерии (одинаковые ``патчи'', реализованные параллельно), а также в биологии (повторные мутации в разных популяциях)~\cite{Rokas2008}.  
Ранняя (парсимонийная) техника анализа обычно фиксирует минимальное число изменений, не давая количественной меры, отражающей степень параллельности.  
Проблема усложняется, если число различных ветвей велико, и требуется формализованная методика с ранжированием по ``важности''. 

Биологические приложения занимают особое место в перечисленных задачах.  
Эволюционные изменения геномов можно рассматривать как особый вид информационных процессов, в ходе которых происходят сложные преобразования структурной организации генетической информации. 
Количество таких операций изменения между двумя видами часто служит мерой эволюционной дистанции между ними. 
При сравнении и эволюции геномов блоки (гены) можно рассматривать как перестановки, и расстояние между ними (количество инверсий или транспозиций) даёт оценку эволюционной близости~\cite{yancopoulos2005,braga2010}.  
При описании взаимодействий генов или клеточных состояний удобно использовать случайные графы, причём требуются модели, учитывающие различную ``интенсивность'' связей~\cite{Barabsi2004}. 
Такие обобщённые модели (с аффинностями) могут предсказывать появление ``гигантской компоненты'' при иных порогах, чем классическая модель Эрдёша--Реньи. 
Это существенно влияет на интерпретацию биологических данных, когда слишком упрощённая модель недооценивает или переоценивает вероятность ``слияния'' крупных фрагментов в эволюционном процессе~\cite{tannier2016}.  

Таким образом, актуальными и востребованными \textbf{задачами} являются:  
\begin{enumerate}
    \item разработка и анализ процессов случайных операций над дискретными структурами с неоднородными вероятностями;  
    \item оценка расстояния между конфигурациями с возможностью достоверно учитывать крупные масштабы изменений;  
    \item автоматизации анализа параллельных изменений на деревьях и введении количественных метрик степени их независимого возникновения. 
\end{enumerate}

\textbf{Степень разработки проблемы.}  
Разнообразные аспекты сравнения и эволюции дискретных структур были изучены в ряде фундаментальных и прикладных исследований.  

Случайные графы и их обобщения. 
Классическая модель Эрдёша--Реньи, в процессе измениния которой каждое ребро добаляется с одинаковой вероятностью, нашла широкое применение, описаное, в частности, в работах А.М.~Райгородского~\cite{райгородский2010модели,райгородский2022модели}.  
Позднее было показано, что во многих реальных сетях (социальных, биологических) важно учитывать неоднородность ``аффинностей'' вершин~\cite{tannier2016}.
Данные обобщения позволяют точнее описывать системы с дифференцированным вкладом узлов.
Однако итоговые формулы (например, для порога появления гигантской компоненты) сложны в вычеслении и применении и требуют новых комбинаторных и аналитических результатов~\cite{tannier2016}.  

Сравнение перестановок и вычисление расстояний.
Для описания процесса изменения последовательностей (в том числе геномных) широко применяются метрики на перестановках.
Уже в 1990-х были сформулированы методы вычисления расстояния перестановок (например, через минимальное число операций инверсии/транспозиции)~\cite{yancopoulos2005,braga2010}, а также предложены статистические модели случайных перестроек (DCJ-модель, модель ``хрупких'' регионов)~\cite{Pevzner03,tannier2016}.
Тем не менее, существующие подходы нередко опираются на бесконечные рядовые разложения и трудоёмкие итерационные алгоритмы, которые становятся неустойчивыми при большом количестве изменений~\cite{tannier2016}.
Это затрудняет оценку истинной дистанции и требует поиска новых аналитических решений.  

Обнаружение параллельных изменений на деревьях.
В филогенетическом анализе, а также при изучении версий ПО, культурных традиций и других ``древовидных'' процессах, давно известно, что один и тот же признак (исправление фрагмента кода, мутация, вставка текста и т.~п.) может возникать неоднократно и независимо.  
Методы парсимонии (например, алгоритм Фитча) выявляют минимальное число таких изменений, но не дают количественной меры параллельности~\cite{Avdeyev2016}.
Ранние решения были фрагментарными и использовались, в основном, вручную, когда исследователь сам отмечает ``зоны повторного возникновения''. 
Строгое формальное описание и автоматизация подобного анализа остаются открытой проблемой, особенно при больших масштабах данных.  

Таким образом, к настоящему моменту накоплен значимый теоретический и прикладной инструментарий для исследования случайных дискретных структур, оценки расстояний и анализа эволюционных деревьев.  
Однако существенные ограничения всё ещё сохраняются:

\begin{itemize}
    \item Неоднородность вероятностей далеко не всегда учитывается в традиционных моделях (например, классической модели Эрдёша--Реньи). При этом реальные системы (биологические, социальные) часто требуют более гибких параметров;  
    \item Вычислительная сложность и расходимость рядов в существующих вероятностных моделях для перестановок и графов затрудняют получение точных оценок расстояния при больших масштабах изменений;  
    \item Отсутствие формализованных алгоритмов выявления и количественной оценки параллельных изменений на деревьях: минимальное объяснение парсимонии не отражает ``степень'' и распределённость независимых появлений признака.  
\end{itemize}

Всё это указывает на необходимость разработки новых алгоритмов обработки данных, позволяющих (1) строить обобщённые случайные модели с учётом неоднородных вероятностей, (2) выводить аналитические формулы для расчёта расстояний, преодолевающие проблемы бесконечных рядов, и (3) автоматизировать обнаружение параллельных изменений с количественной оценкой их ``независимости''.  
Результаты таких исследований востребованы как в теоретической математике (расширение классических моделей и методов комбинаторного анализа), так и в прикладных исследованиях, особенно в задачах эволюционной биологии, но и за её пределами --- в лингвистике, анализе версий ПО, культурно-исторических исследованиях и других сферах.  

\textbf{Научная новизна} состоит в том, что: 
(1) впервые получены аналитические выражения для оценки числа компонент в рамках случайных графов с индивидуальными вероятностями (обобщение модели Эрдёша--Реньи), устраняющие необходимость численного суммирования расходящихся рядов.
(2) найден порог появления гигантской компоненты в модели случайных графов с неравномерными аффинностями, что вдвое меньше порога в классической модели Эрдёша–Реньи.
(3) разработан метод оценки истинного расстояния между двумя конфигурациями с учётом неоднородностей, позволяющий устойчиво вычислять метрику при высоком уровне перестроек. 
(4) предложен алгоритм автоматического выявления параллельных изменений на деревьях. 
Введена новая комбинаторная метрика параллельности, позволяющая ранжировать независимые события по степени их распределённости на разных ветвях.


\textbf{Теоретическая значимость} определяется расширением классических вероятностных постановок путём введения неоднородных вероятностей, а также количественной формализацией параллельных изменений на деревьях.  
В частности: (1) получены новые комбинаторные и асимптотические результаты, описывающие ожидаемое количество компонент заданного размера и появление гигантской компоненты для графов с индивидуальными аффинностями вершин; 
(2) предложены метод оценки расстояний между перестановками при больших масштабах изменений;
(3) систематизирован подход к ранжированию случаев независимых изменений в древовидной топологии.

\textbf{Практическая значимость работы} определяется:

\begin{enumerate}
    \item Повышение точности оценки расстояний при больших больших масштабах изменений, что важно для сравнительного анализа геномов, крупных текстовых данных.
    \item Автоматизированная идентификация и ранжирование параллельных (независимых) изменений, востребованная в биоинформатике (выявление конвергентных мутаций), лингвистике (одинаковые инновации в родственных языках) и др.
    \item Программная реализация (пакет \emph{TruEst} для вычисления расстояний и \emph{PaReBrick} для обнаружения параллельных событий), открытая для интеграции в другие исследовательские инструменты.
\end{enumerate}

\textbf{На защиту выносятся положения, обладающие научной новизной:}
\begin{enumerate}[label={\arabic*.}]
    \item Комбинаторный метод описания процесса изменения случайных графов с неравномерными аффинностями (обобщающий классическую модель Эрдеша-Реньи), отличающийся тем, что, с целью корректного учёта неоднородных вероятностей рёбер, предложены аналитические формулы для оценки числа компонент связности заданного размера и доказан новый порог появления гигантской компоненты, что расширяет применимость модели.
    \item Алгоритм оценки расстояния между объектами, представленными перестановками на основе случайных графов с неоднородными вероятностями состояний, отличающийся тем, что, с целью повышения точности вычислений на больших расстояниях, вместо численного суммирования потенциально расходящихся рядов используются аналитические выражения для ключевых характеристик циклограммы перестановки, что позволило реализовать устойчивое вычисление расстояния даже при высоком уровне эволюционных изменений.
    \item Алгоритм выявления и ранжирования независимых изменений в процессах эволюции наборов перестановок на древовидных информационных структурах, отличающийся тем, что, с целью автоматического и объективного обнаружения повторяющихся (конвергентных) событий в информационных процессах, вводится новая комбинаторная метрика — «показатель параллельности», количественно отражающая как частоту и количество независимых изменений, так и их распределённость по вершинам дерева, что повышает достоверность и наглядность анализа параллельных эволюционных изменений.
\end{enumerate}

\textbf{Методы исследования.} 
В работе использованы методы теории вероятностей и математической статистики, комбинаторные методы и алгоритмы на деревьях, методы численной оптимизации и анализа сходимости, экспериментальные тесты на синтетических и реальных данных (в первую очередь, геномных), оценивающие точность и скорость разработанных алгоритмов.

\textbf{Достоверность} научных результатов обеспечена: строгими математическими доказательствами корректности полученных формул, валидацией на симулированных данных, где истинные параметры известны заранее, сравнением с опубликованными результатами и моделями (включая классические алгоритмы оценки расстояния по перестановкам), открытым доступом к программному коду (\texttt{GitHub}-репозитории \emph{TruEst} и \emph{PaReBrick}), позволяющим независимо воспроизвести эксперименты.

\textbf{Соответствие паспорту специальности.}
Полученные научные результаты соответствуют следующим пунктам паспорта специальности 2.3.8~--- ``Информатика и информационные процессы (технические науки)'':

\textbf{Пункт 1}~--— ``Разработка компьютерных методов и моделей описания, оценки и оптимизации информационных процессов и ресурсов, а также средств анализа и выявления закономерностей на основе обмена информациейпользователями и возможностей используемого программно-аппаратного обеспечения''.
Разработаны компьютерные методы описания и оптимизации информационных процессов изменения дискретных структур (перестановок и графов) с учётом неоднородных вероятностей.;

\textbf{Пункт 4}~--— ``Разработка методов и технологий цифровой обработки аудиовизуальной информации с целью обнаружения закономерностей в данных, включая обработку текстовых и иных изображений, видео контента. Разработка методов и моделей распознавания, понимания и синтеза речи, принципов и методов извлечения требуемой информации из текстов.''.
Разработаны методы и алгоритмы обработки и анализа информации, включая автоматизированный анализ параллельных изменений.

% \textbf{Пункт 7}~--— ``Разработка методов обработки, группировки и аннотирования информации, в том числе, извлеченной из сети интернет, для систем поддержки принятия решений, интеллектуального поиска, анализа''.
% Предложены алгоритмы автоматизированной обработки и группировки информации, полученной из различных источников.

\textbf{Апробация результатов работы}

Основные результаты работы были представлены на следующих  конференциях:

\begin{itemize}
    \item RECOMB Comparative Genomics, 2022, онлайн;
    \item XI Конгресс молодых ученых, 2022, Университет ИТМО, Санкт-Петербург, Россия;
    \item Moscow Conference on Computational Molecular Biology, 2021, Москва, Россия;
    \item Пятидесятая научная и учебно-методическая конференция, 2021, Университет ИТМО, Санкт-Петербург, Россия;
    \item BiATA 2020 (Bioinformatics: From Algorithms to Applications), 2020, онлайн;
    \item RECOMB Comparative Genomics (постерный доклад), 2019, Монтпелье, Франция;
    \item Вероятностные методы в дискретной математике, 2019, Петрозаводск, Россия;
    \item Moscow Conference on Computational Molecular Biology, 2019, Москва, Россия;
    \item RECOMB Comparative Genomics, 2018, Шербрук, Канада;
\end{itemize}