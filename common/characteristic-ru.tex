%!TEX root = ../dissertation.tex

\nociteallauthorpublicationsmaindocument

\textbf{Актуальность темы исследования.} 
Модели метрических деревьев с функциями на ребрах применяются для анализа и прогнозирования различных явлений реального мира, например, процессов, представимых в виде динамических систем с переменной структурой~\cite{кириллов2009динамические, aldous1993continuum}.
Под \textit{метрическими деревьями} (metric trees) понимают граф, являющийся деревом, где каждому ребру поставлен в соответствие интервал.
В общем виде, метрические графы с функциями на ребрах нашли широкое применение, например, в виде квантовых графов~\cite{berkolaiko2013introduction}, которые используются в физике при изучении квантового хаоса~\cite{kottos1997quantum}, волноводов~\cite{exner2015quantum} и фотонных кристаллов~\cite{kuchment2002differential}.

\textit{Построение модели} представляет собой набор действий, направленных на выбор конфигурации, определение параметров модели и настройку их значений с целью достижения высокого соответствия результатов моделирования данным натурного эксперимента.
На различных этапах построения модели часто требуются экспертные данные или предположения об исследуемом объекте. Эти данные могут быть неточными, ограниченными или неизвестными, что может негативно сказаться на точности и адекватности модели.
Методы автоматизированного построения позволяют уменьшить вероятность человеческих ошибок при выборе модели и настройке ее параметров.

При работе с моделями метрических деревьев с функциями на ребрах прибегают к участию предметных специалистов.
В первую очередь экспертные данные используются для определения свойств функций на ребрах дерева.
Эта информация позволяет установить \textit{конфигурацию} модели, где каждая функция на дереве принадлежит заданному семейству и характеризуется функциональными параметрами, доступными для настройки.
В условиях отсутствия экспертных данных или для минимизации влияния специалиста на получение результата приходится рассматривать множество всех возможных моделей, отличающихся типами функций и функциональными параметрами.
Например, при построении моделей демографических историй для каждой популяции в качестве динамики изменения численности обычно рассматриваются кусочно-заданные функции, состоящие из функций трех наиболее популярных типов: константная, линейная и экспоненциальная.
Такой перебор конфигураций приводит к увеличению временных затрат при построении модели, тем больших, чем больше допустимых типов функций.
Дополнительно, требуется следить за сложностью модели, числом ее параметров и переобучением.

Методы для настройки параметров моделей также могут быть ограничены в степени автоматизации и требовать экспертных данных.
Например, при использовании методов локального поиска требуются вовлечение специалиста для определения начальных значений параметров, и эффективность настройки зависит от этого выбора.

\newpage
Таким образом, при моделировании явлений реального мира в виде метрического дерева с функциями на ребрах \textit{актуальна} разработка специализированных моделей и методов для автоматического построения и настройки моделей с целью минимизации влияния экспертных данных на результат моделирования, что рассматривается в данной диссертации на примере задачи вывода демографических историй по генетическим данным.

Популяция --- это группа особей одного вида, живущих на одной территории.
\textit{Демографическая история популяций} --- это исторический процесс их развития и эволюции, который включает в себя такие явления, как изменения численности популяций, разделения популяций, миграция и отбор.
%Используя различные статистические и алгоритмические методы, возможно ее реконструировать или вывести по генетическим данным.
Демографические истории используются для датирования исторических событий, не оставивших письменных свидетельств~\cite{goebel2008late,mellars2006going}, а также играют важную роль в области консервативной генетики~\cite{nikolic2022stepping} и даже в медицине~\cite{nielsen2007recent}.

Различные статистические и алгоритмические методы позволяют строить модели демографических историй в виде метрических деревьев с функциями на ребрах и настраивать их непрерывные параметры по генетическим данным.
В случае демографических историй, метрическое дерево является деревом, которое определяет разделение популяций, а функции на ребрах — динамиками изменения численности популяций.
В качестве динамик рассматривают кусочно-заданные функции, состоящие из функций трех наиболее популярных типов: константная, линейная и экспоненциальная.
При построении моделей требуется определить число временных интервалов, а также тип динамики для каждой кусочно-заданной функции.

Вовлечение специалиста также происходит и на этапе настройки параметров моделей демографической истории популяций, для чего используются комбинация методов численного моделирования и методов оптимизации.
Методы численного моделирования используются для вычисления функции правдоподобия, которая позволяет оценить степень соответствия модели генетическим данным.
Для поиска параметров, обеспечивающих максимальное значение правдоподобия, используются методы локальной оптимизации. Именно эти методы ограничены в степени автоматизации: они требуют экспертных данных для определения начальных значений параметров, а их эффективность зависит от этого выбора.

Задача вывода демографической истории популяций дополнительно усложняется необходимостью реализации пользователем программного кода модели и алгоритма вывода ее параметров.
Методы численного моделирования, используемые существующими решениями, имеют разные возможности и стабильность работы, и пользователь может применить несколько из них для сравнения результатов.
Однако при применении различных программных решений одновременно, пользователь сталкивается с необходимостью задавать одни и те же модели с использованием разных интерфейсов.

Таким образом, развитие методов автоматического построения и настройки метрических деревьев с функциями на ребрах приведет к минимизации влияния экспертных данных, и, следовательно, к повышению качества моделирования явлений реального мира по данным натурного эксперимента.\\

%Несмотря на наличие методов вывода демографической истории, выбор модели и начальных значений ее параметров все еще требуют ручной настройки и экспертных данных, что ограничивает методы получения результата.
%Для улучшения точности и достоверности вывода пользователь вынужден рассматривать множество возможных моделей, находить оптимальные значения параметров и сравнивать результаты между собой.
%Рассматриваемые модели, в первую очередь, отличаются динамикой изменения численности популяций, тип которой всегда зафиксирован в модели.
%При этом обычно используют три наиболее популярных динамики: константная численность, линейное и экспоненциальное изменения.
%Неверный выбор начальных оценок параметров моделей может снизить эффективность используемого алгоритма оптимизации и привести к недостоверным результатам при сравнении моделей.


%Таким образом, для построения и автоматической настройки моделей метрических графов с функциями на ребрах необходимо использовать специализированные методы моделирования и вычислительной оптимизации, что и рассматривается в диссертации на примере моделей демографических историй.\\


\textbf{Степень разработки проблемы.}
Модели графов исследуются и применяются для решения широкого круга задач.
В работах А.М.~Райгородского~\cite{райгородский2010модели,райгородский2022модели} приведены описания и примеры применения моделей случайных графов.
Графовые вероятностные модели, такие как байесовские сети, обширно представлены в работах И.~Бена-Гала~\cite{ben2008bayesian} для моделирования индустриальных систем~\cite{gruber2012efficient}, классификации~\cite{gruber2019targeted} или идентификации сайтов связывания транскрипционных факторов~\cite{ben2005identification}.
Л.~Кларк и Д. Прегибон~\cite{clark2017tree} описали примеры применения моделей, основанных на деревьях, к которым относятся, например, решающие деревья~\cite{kotsiantis2013decision}.

Теория метрических графов была сформирована работами В.Г.~Болтянского~\cite{болтянский1978комбинаторная}, П.С.~Cолтана~\cite{soltan1973экстремальные,болтянский1978комбинаторная} и А.~Дресса~\cite{dress1984trees}.
Свойства метрических деревьев и метрических пространств, порожденных ими, были изучены А.~Дрессом~\cite{dress1984trees}, Б.~Бунеманом~\cite{buneman1974note} и Д.~Олдосом~\cite{aldous1991continuum_i,aldous1991continuum,aldous1993continuum}.
В работах А.С.~Матвеева и С.И.~Матвеева~\cite{матвеев2010создание,лёвин2011системы,матвеев2013интеллектуальная} метрические графы были применены при построении координационных моделей для интеллектуальной навигации.

Разработкой моделей, приближающих неявные функции, также активно занимаются многие ученые.
Наиболее широкое применение, описанное в работах Л.~Фармейра~\cite{fahrmeir2013regression} и Р.~Сни~\cite{snee1977validation}, эти модели получили для решения задач регрессии.
При использовании моделей кусочно-заданных функций обычно фиксируют общий вид формирующих функций, например, строят кусочно-постоянные~\cite{schiffels2020msmc,dai2008nonlinear}, кусочно-линейные~\cite{leenaerts2013piecewise} или кусочно-экспоненциальные~\cite{friedman1982piecewise} модели.
Число точек смены функции, а также их положение являются неизвестными характеристиками моделей кусочно-заданных функций.
В работах~\cite{muggeo2020selecting,malash2010piecewise} рассмотрены методы автоматического построения таких моделей для решения задачи кусочно-заданной регрессии, где число точек смены функции определяется с использованием байесовского информационного критерия (BIC) и информационного критерия Акаике (AIC)~\cite{akaike1974new} соответственно.

Модели метрических деревьев с функциями на графах являются комбинацией моделей метрических деревьев и функциональных моделей на ребрах.
Квантовые графы, которые являются метрическими графами с дифференциальными операторами на ребрах, и их приложения подробно рассмотрены в работах Г.~Берколайко~\cite{berkolaiko2006quantum,berkolaiko2013introduction}.
Метрические деревья с функциями на ребрах используются для моделирования демографических историй популяций в работах Р.~Гутенкунста~\cite{gutenkunst2009inferring}, Д.~Камма~\cite{kamm2020efficiently}, А.~Рэгсдейла и С.~Гравеля~\cite{ragsdale2019models, ragsdale2020unbiased}.
Однако методы, представленные в этих работах, предполагают, что пользователь определяет и фиксирует общий вид кусочно-заданной функции на ребрах дерева, а также задает начальные значения параметров настройки параметров методами локальной оптимизации.
В работах Д.~Портика~\cite{portik2017evaluating, leache2019exploring} и Р.~Гутенкунста~\cite{blischak2020inferring} были представлены методы глобальной оптимизации для настройки параметров моделей демографических историй, которые минимизируют, однако все еще требуют вовлечение пользователя.
Общее применение методов численной оптимизации для решения задач представлено в классической работе Б.Т. Поляка~\cite{поляк1983введение}, а описание современных методов глобальной оптимизации в работе~\cite{пантелеев2013методы}.


На момент начала исследований автором (в 2017 году) не существовало метода автоматического построения и настройки моделей метрических деревьев с функциями на ребрах.
К концу диссертационного исследования появилось первое альтернативное решение для метода автоматического перебора моделей на примере задачи вывода демографических историй~\cite{rippe2021environmental}.
Однако метод позволяет анализировать модели, определенные специфичным каталогом и только для вывода демографической истории \emph{двух} популяций, а выбор наилучшей модели происходит в предположении независимости данных, что не всегда является корректным.\\

%Таким образом, в настоящее время всё ещё не предложены подходы, позволяющие осуществлять автоматическое построение и настройку моделей метрических графов с функциями на ребрах.\\

% Основы методов вывода демографической истории популяций заложили японский биолог М.~Кимура в работах 1962~\cite{kimura1962probability,kimura1964diffusion} и 1969~\cite{ohta1969linkage} годов, а также ученые В.~Хилл и А.~Робертсон в работах 1966~\cite{hill1966effect} и 1968~\cite{hill1968linkage} годов.
% Однако только с развитием методов секвенирования и с накоплением генетических данных эти методы стали активно применяться.
% В конце XX века были разработаны методы численного моделирования для оценки отдельных характеристик демографических историй, например, степени роста численности популяции~\cite{kuhner1998maximum}.

% Затем в начале XXI века стали появляться методы и программные комплексы для построения и настройки моделей метрических графов с функциями на ребрах для вывода демографических историй.
% Среди наиболее популярных программных средств можно назвать библиотеку \dadi~\cite{gutenkunst2009inferring}, реализующую метод аппроксимации диффузией для вычисления правдоподобия, библиотеку \momi~\cite{kamm2020efficiently}, а также библиотеки \moments~\cite{jouganous2017inferring} и \momentsLD~\cite{ragsdale2019models, ragsdale2020unbiased}, применяющие метод моментов.
% %Все эти методы являются методами численного имитационного моделирования.
% %Популярные библиотеки для настройки параметров моделей метрических графов с функциями на ребрах для вывода демографических историй представлены в таблице~\ref{tab:synopsis:list_dem_methods}.

% %Отметим, что сложность методов вычисления правдоподобия растет с увеличением рассматриваемого числа популяций.
% %Как следствие, некоторые программные средства поддерживают только ограниченное число популяций для анализа.
% %Так, например, \dadi и \moments имеют экспоненциальную сложность методов вычисления правдоподобия и позволяют анализировать только до трех  и пяти популяций соответственно.

% % \begin{table}[ht]
% %     \centering
% %     \resizebox{\linewidth}{!}{%
% %     \begin{tabular}{|l|l|l|l|l|l|l|}
% %         \hline
% %         Программное & Год & Интерфейс для &  Метод & Методы        & Требуются & Число \\
% %         средство    &     & cпецификации & вычисления    & оптимизации  & начальные & популяций \\
% %                    &     & моделей & правдоподобия &              &                  параметры & \\
                   
% %         \hline
% %         \dadi   & 2009  & Да & Аппроксимация& Четыре метода & Да & До трех \\
% %                 &       & (модели & диффузией    & локальной  & & \\
% %                 &       & I класса) &              & оптимизации      & & \\
% %                 &       & &              & плюс один метод      & & \\
% %                 &       & &              & глобальной      & & \\
% %                 &       & &              & оптимизации (2020 г.)     & & \\
% %         \hline
% %         \moments& 2017  & Да & Метод моментов& Четыре метода &  Да & До пяти \\
% %                 &      & (модели & для статистики  & локальной  & & \\
% %                 &   &  I класса) & частоты аллелей & оптимизации      & & \\
% %         \hline
% %         \momentsLD& 2019 & Да & Метод моментов& Четыре метода & Да & Произвольное \\
% %                 &       &  (модели & для статистик & локальной  & & \\
% %                 &      & I класса) & неравновесного & оптимизации      & & \\
% %                 &      & & сцепления генов &       & & \\
% %         \hline
% %         \momi   & 2020  & Да & Непрерывная    & Один метод & Да & Произвольное \\
% %                 &   & (модели    & модель Морана  & усеченный & & \\
% %                 &   &   II класса) &                & метод & & \\
% %                 &    &   &                & Ньютона       & & \\
% %         \hline
% %         \textit{dadi-pipeline}   & 2017  & Нет & Метод из \dadi  & Один метод & Нет & До трех \\
% %                 &   & (интерфейс    &   & множественного & &\\
% %                 &   &   \dadi) &                & запуска метода & & \\
% %                 &    &    &                & Нелдера-Мида       & & \\
% %         \hline
% %         \textit{moments-pipeline}   & 2019  & Нет & Метод из \moments  & один метод & Нет & до пяти \\
% %                 &   & (интерфейс     &   & множественного & &\\
% %                 &   &   \moments) &                & запуска метода & & \\
% %                 &    &    &                & Нелдера-Мида       & & \\
% %         \hline

% % %        \textit{fastsimcoal2} & 2013 
% % %          & Симуляция    & 1 алгоритм  & до $\infty$ &  Да & Да\\
% % %        & & процесса     & ECM-алгоритм & & & \\
% % %        & & коалисценции & (Expectation & & & \\
% % %        & &              & Conditional & & & \\
% % %        & &              & Maximization) & & & \\
% % %        \hline
% %     \end{tabular}%
% %     }
% %     \caption{Существующие программные средства для вывода демографической истории популяций по генетическим данным}
% %     \labelsyn{tab:synopsis:list_dem_methods}
% % \end{table}

% На момент начала исследований автором (в 2017 году) не существовало алгоритма поиска демографической истории популяций без заданной конфигурации модели, а также методов глобальной оптимизации для эффективного поиска ее параметров.
% Все существующие решения требовали заданную исследователем модель, а также начальную оценку ее параметров для алгоритмов локального поиска, что представляло трудность для основной аудитории этих программных средств и приводило к ошибкам использования.
% Более того, поиск параметров модели для большого числа популяций, как, например, четыре или пять популяций для библиотеки \moments, представлял большие трудности, так как требовал значительных вычислительных затрат.

% В 2017 и 2019 годах в работах Д.~Портика были представлены программные средства \textit{dadi pipeline} и \textit{moments pipeline} как оболочки для библиотек \dadi и \moments соответственно~\cite{portik2017evaluating, leache2019exploring}.
% Пользователь может выбирать реализованные модели из предоставленного каталога, что не решает задачу автоматического перебора моделей, но устраняет необходимость их задания вручную и, следовательно, уменьшает вероятность ошибки.
% Кроме того, в \textit{dadi pipeline} и \textit{moments pipeline} был реализован алгоритм глобального поиска, который последовательно запускает несколько раундов локальной оптимизации Нелдера-Мида.
% Для решения задачи выбора начальных параметров был предложен подход, который включает в себя значения параметров, единых для всех моделей, и применение случайных колебаний к значениям параметров перед каждым раундом оптимизации.
% Таким образом, предложенный метод содержит несколько гиперпараметров: число раундов алгоритма локального поиска, а также амплитуда колебаний, применяемый к значениям параметров.
% Важно отметить, что оптимальные значения этих гиперпараметров, необходимые для достижения глобального оптимума, не были изучены и оставались на усмотрение пользователя.

% %Первая публикация диссертанта~\cite{noskova2020gadma} в 2020 году была посвящена проблемам неэффективности существующих оптимизаций и отсутствию метода для автоматического поиска конфигурации.
% %В работе были предложены расширенный класс моделей, генетический алгоритм для настройки параметров и метод автоматического перебора моделей, описанные в данной диссертации.
% %После предварительной нерецензируемой публикации этой статьи и отправки ее в журнал в 2019 году авторы библиотеки \dadi включили в библиотеку алгоритм глобального поиска BOBYQA~\cite{powell2009bobyqa}, который, однако, все также требует от пользователя задания начальной оценки параметров~\cite{blischak2020inferring}.

% % Через год после публикации статьи~\cite{noskova2020gadma} появилось первое альтернативное решение для метода автоматического поиска модели демографической истории \emph{двух} популяций~\cite{rippe2021environmental}.
% % Предложенное программное обеспечение является оболочкой для \moments и представляет собой перебор моделей из большого каталога с более чем 100 моделями, настройку их параметров и сравнение.
% % Выбор наилучшей модели происходит в предположении независимости данных, что не всегда является корректным.
% % В последней версии этого программного средства в качестве алгоритма оптимизации используется генетический алгоритм, разработанный в данной работе.

% % Появление новых методов выбора модели и алгоритмов оптимизации для задачи вывода демографической истории популяции, а также активное использование разработанных диссертантом методов показывает повышенный интерес к задаче, решаемой в данной работе, и делает ее актуальной.\\

% К концу диссертационного исследования авторы библиотеки \dadi включили в библиотеку алгоритм глобального поиска BOBYQA~\cite{powell2009bobyqa}, который, однако, все также требует от пользователя задания начальной оценки параметров~\cite{blischak2020inferring}.
% Также появилось первое альтернативное решение для метода автоматического поиска модели демографической истории, разработанного в данной работе~\cite{rippe2021environmental}.
% Однако предложенное программное обеспечение позволяет анализировать данные только \emph{двух} популяций с использованием \moments.
% Выбор наилучшей модели происходит в предположении независимости данных, что не всегда является корректным.
% В последней версии этого программного средства в качестве алгоритма оптимизации используется генетический алгоритм, разработанный в данной работе.\\

\textbf{Целью} настоящей диссертации является повышение качества\footnote{Качество моделей в данной работе определяется степенью соответствия настроенной модели данным натурного эксперимента. В случае задачи вывода демографических историй популяций качество определяется значением функции правдоподобия, полученным численными методами за фиксированное время настройки модели.}
компьютерного моделирования явлений реального мира за счет автоматизации построения и настройки моделей метрических деревьев с функциями на ребрах.\\

Для решения цели в диссертации решаются следующие \textbf{задачи}:

\begin{itemize}
    \item исследование текущего состояния предметной области, уточнение задачи и способов оценки результатов;
    \item формализация постановки задачи построения и настройки моделей метрического дерева с функциями на ребрах;
    \item разработка метода автоматической настройки моделей метрического дерева с функциями на ребрах на основе комбинации методов глобальной и локальной оптимизации;
    \item разработка метода автоматического перебора моделей метрического дерева с кусочно-заданными функциями на ребрах;
    %\item разработка расширенного класса моделей, который включает модели не только с непрерывными параметрами, но и с дискретными параметрами динамики изменения численности;
    %\item разработка методов настройки параметров моделей расширенного класса на основе комбинации методов глобальной и локальной оптимизации;
    %\item разработка метода автоматического перебора моделей расширенного класса;
    \item проектирование и реализация программного комплекса, включающего разработанные модели и методы для вывода демографической истории популяций по генетическим данным;
    %\item расширение существующих программных средств для совместного использования, проведения экспериментальных исследований и корректного сравнения;
    %\item экспериментальные исследования для выявления эффективности разработанных моделей и методов на симулированных и реальных данных;
    \item проведение экспериментальных исследований, подтверждающих эффективность разработанных моделей и методов, а также их применимость для вывода демографической истории популяций по генетическим данным, анализ результатов экспериментов.\\
    %\item оценка качества предложенных методов путем сравнения результатов экспериментальных исследований с существующими решениями;
    %\item применение разработанных методов для вывода демографических историй популяций по реальным генетическим данным, которые ранее не были проанализированы.\\
\end{itemize}

\textbf{Научная новизна} диссертации состоит в том, что: 
%(1) разработан расширенный класс моделей, который включает модели с дискретными параметрами динамики изменения численности популяций для настройки;
(1) разработаны методы на основе комбинации методов глобальной и локальной оптимизации для настройки параметров заданной модели метрического дерева с функциями на ребрах;
(2) разработан метод автоматического перебора моделей метрического дерева с кусочно-заданными функциями на ребрах, не требующий вовлечения эксперта на этапе выбора параметров рассматриваемых моделей.\\
%(4) для предметной области получены новые демографические истории популяций, которые лучше описывают генетические данные, чем полученные ранее с помощью методов локального поиска и экспертного участия.\\

\textbf{Теоретическая значимость}
%работы определяется расширением класса моделей демографической истории популяций, а также расширением классической постановки задачи вывода демографической истории популяций не только как задачи настройки параметров заданной модели демографической истории, но и как задачи построения самой модели путем автоматического перебора.
работы определяется расширением классической постановки задачи настройки модели метрического дерева с функциями на ребрах не только как задачи настройки параметров заданной модели, но и как задачи выбора самой модели путем автоматического перебора.
%Полученные методы настройки параметров моделей доступны для совместного применения со всеми существующими, а также с будущими методами численного имитационного моделирования для вычисления правдоподобия.
Полученные методы моделирования и настройки применимы для произвольных моделей метрического дерева с функциями на ребрах.
Более того, разработанные методы оптимизации могут быть использованы или адаптированы для задач поиска оптимальных параметров в других научных областях.\\

\textbf{Практическую значимость} работы определяют:
\begin{itemize}
    \item расширение научно-практического инструментария специалистов-биоинформатиков методами и алгоритмами для вывода демографических историй популяций;
    \item открытый программный код разработанного программного комплекса GADMA, который доступен к переиспользованию по адресу \url{https://github.com/ctlab/GADMA};
    \item применимость разработанных методов для анализа генетических данных;
    \item внедрение разработанного метода на основе генетического алгоритма в стороннее программное решение~\cite{rippe2021environmental}.\\
\end{itemize}

%\newpage
\textbf{На защиту выносятся положения, обладающие научной новизной:}
\begin{enumerate}[label={\arabic*.}]
    \item Метод моделирования и настройки параметров  моделей метрических деревьев с функциями на ребрах по данным натурного эксперимента, содержащий модели с непрерывными функциональными параметрами, отличающийся тем, что с целью автоматической настройки без привлечения экспертных данных в нем используются модели с дискретными парамерами, определяющими семейства функций, а также методы глобальной оптимизации --- генетический алгоритм и байесовская оптимизация, и реализующий его комплекс программ.
    \item Метод автоматического перебора моделей метрических деревьев с функциями на ребрах с разным числом параметров и настройки этих параметров по данным натурного эксперимента, содержащий сравнение моделей с использованием информационного критерия Акаике, отличающийся тем, что с целью повышения уровня автоматизации  и обеспечения возможности настраивать не только параметры модели, но и саму модель, он включает метод увеличения числа временных интервалов для кусочно-заданных функций на ребрах дерева, а также реализующий его комплекс программ.\\
    %\item Расширенный класс моделей демографической истории популяций, содержащий модели с непрерывными параметрами, отличающийся тем, что с целью повышения удобства пользователя, он дополнительно включает модели с дискретными параметрами динамики изменения численности популяций.
    %\item Метод настройки параметров расширенных моделей по заданным генетическим данным, содержащий методы численного моделирования, отличающийся тем, что с целью поиска демографической истории с наибольшим значением правдоподобия в нем используются методы глобальной оптимизации --- генетический алгоритм и байесовская оптимизация.
    %\item Метод автоматического перебора расширенных моделей с разным числом параметров и настройки их параметров по генетическим данным одной, двух и трех популяций, отличающийся тем, что с целью повышения уровня автоматизации  и обеспечения возможности настраивать не только параметры модели, но и саму модель демографической истории, он включает метод увеличения числа временных интервалов модели.
    %\item Программный комплекс для поиска демографической истории популяций по генетическим данным, содержащий существующие методы вычисления функции правдоподобия, отличающийся тем, что с целью повышения уровня автоматизации и уменьшения требуемых знаний о модели от конечного пользователя, он дополнительно содержит методы глобальной оптимизации и метод автоматического перебора моделей.\\
    %Программный комплекс для поиска демографической истории популяций по генетическим данным, отличающийся повышенным уровнем автоматизации и требующим меньших экспертных знаний о модели от конечного пользователя.
\end{enumerate}

\textbf{Методы исследования.} В работе использованы методы оптимизации, численные методы, методы теории вероятности и математической статистики, методы машинного обучения и методы проведения экспериментальных
исследований.\\

\textbf{Достоверность} научных результатов обусловлена корректным использованием методов, обоснованием постановки задач, экспериментальными исследованиями, покрывающими разработанные технологии и алгоритмы.
Демографические истории, полученные разработанными методами на проверяемых симулированных данных, согласуются с исходными историями, используемыми для моделирования. Результаты, полученные на реальных данных, согласуются с опубликованными ранее исследованиями~\cite{gutenkunst2009inferring, jouganous2017inferring, nielsen2017tracing, verissimo2017world, king2015genetic, сивцева2020геном}.\\

\textbf{Соответствие паспорту специальности.}
Полученные научные результаты соответствуют следующим пунктам паспорта специальности 1.2.2 --- «Математическое моделирование, численные методы и комплексы программ (технические науки)».

\textbf{Пункт 2 паспорта специальности} «Разработка, обоснование и тестирование эффективных вычислительных методов с применением современных компьютерных технологий».
Были разработаны, обоснованы и протестированы методы настройки параметров моделей метрического дерева с функциями на ребрах, основанные на методах численной оптимизации.
%Были разработаны, обоснованы и протестированы два эффективных вычислительных метода: 1) метод настройки параметров расширенных моделей демографической истории по заданным генетическим данным, 2) метод автоматического перебора моделей демографической истории.


\textbf{Пункт 4  паспорта специальности} «Разработка новых математических методов и алгоритмов интерпретации натурного эксперимента на основе его математической модели».
В диссертационном исследовании представлены методы для построения моделей метрического дерева с функциями на ребрах по данным натурного эксперимента с целью анализа явлений реального мира.\\
%В диссертационном исследовании представлен расширенный класс моделей демографической истории популяций.
%Разработанные методы моделируют демографическую историю популяций по генетическим данным (данные натурного эксперимента).

%\textbf{Пункт 9  паспорта специальности} «Постановка и проведение численных экспериментов, статистический анализ их результатов, в том числе с применением современных компьютерных технологий (технические науки)».
%Разработка программного комплекса GADMA позволила провести множество вычислительных экспериментов по выводу демографических историй популяций по генетическим данным.
%Результаты проведенных вычислительных экспериментов были проанализированы, различные модели были сравнены с использованием статистических методов таких, как информационный критерий Акаике и тест отношения правдоподобий.\\


% В паспорте специальности указано:\\
% \textit{*Диссертационное исследование должно содержать все три составляющих названия специальности}

% В работе присутствуют:
% \begin{itemize}
%     \item методы математического моделирования \textbf{в части} новой модели и методов ее моделирования на основе данных натурного эксперимента;
%     \item численные методы \textbf{в части} используемых методов для вычисления функции правдоподобия, а также \textbf{в части} метода статистического критерия Акаике~\cite{akaike1974new} в условиях зависимости данных;
%     \item комплекс программ \textbf{в части} реализации всех методов в программном комплексе GADMA.\\
% \end{itemize}

\textbf{Апробация результатов работы}

Основные результаты работы были представлены на следующих  конференциях:

\begin{itemize}
    \item Международный конгресс «VII съезд Вавиловского общества генетиков и селекционеров, посвященный 100-летию кафедры генетики СПбГУ, и ассоциированные симпозиумы», 2019, Санкт-Петербург, Россия;
    \item Moscow Conference on Computational Molecular Biology, 2019, Москва, Россия;
    \item Probabilistic Modeling in Genomics, 2019, Осуа, Франция;
    \item Probabilistic Modeling in Genomics, 2021, онлайн;
    \item Moscow Conference on Computational Molecular Biology, 2021, Москва, Россия;
    \item Вероятностные методы в анализе: пространства голоморфных функций, 2021, Сочи, Россия;
    \item LI Научная и учебно-методическая конференция Университета ИТМО, 2022, Университет ИТМО, Санкт-Петербург, Россия;
    \item Probabilistic Modeling in Genomics, 2022, Окфорд, Великобритация;
    \item XI Конгресс молодых ученых, 2022, Университет ИТМО, Санкт-Петербург, Россия;
    \item Conservation Genomics at the Population Level, 2022, Кембридж, Великобритания;
    \item Probabilistic Modeling in Genomics, 2023, Колд Спринг Харбор, США;
    \item XII Конгресс молодых ученых, 2023, Университет ИТМО, Санкт-Петербург, Россия;
    \item Society for Molecular Biology and Evolution Meeting (SMBE23), 2023, Феррара, Италия.\\
\end{itemize}

\textbf{Награды} 
\begin{itemize}
    \item Бронзовая награда в номинации 17th Human-Competitive Awards на онлайн конференции The Genetic and Evolutionary Computation Conference (GECCO) в 2020 году.
    \item Победитель конкурсной программы поддержки исследовательских проектов System Biology Fellowship от Сколковского института науки и технологий по проекту «Computational methods for unsupervised demographic inference of multiple populations from genomic data» в 2021 году. Число победителей --- пять на всю страну в год.\\
\end{itemize}

\textbf{Публикации}

По результатам, представленным в диссертации, было опубликовано восемь статей в рецензируемых научных журналах, входящих в международные реферативные базы данных и системы цитирования Scopus и Web of Science.\\


\textbf{Личный вклад автора}

\begin{enumerate}[label=\arabic*.]
    \item 
    В публикации~\cite{noskova2020gadma} Noskova E. --- разработка и реализация генетического алгоритма и метода автоматического перебора моделей демографической истории, проведение экспериментальных исследований (80\%); Ulyantsev~V. --- рекомендации по постановке задачи, выбору и обоснованию теоретических основ научного исследования (10\%); Koepfli~K.P., O’Brien~S.J. ---  консультирование при проведении экспериментальных исследований и написании статей (5\%); Dobrynin~P. --- рекомендации по постановке задачи (5\%).

    \item %\cite{noskova2022gadma2}: 
    В публикации~\cite{noskova2023gadma2} Noskova E. --- разработка и реализация методов, программного обеспечения для вывода демографической истории популяций по генетическим данным, проведение экспериментальных исследований (85\%); Abramov~N., Iliutkin~S., Sidorin~A. --- разработка программного обеспечения (10\%); Dobrynin~P., Ulyantsev~V. --- рекомендации по постановке задач, выбору и обоснованию теоретических основ научного исследования (5\%).

    \item %\cite{noskova2022bayesian}: 
    В публикации~\cite{noskova2023bayesian} Noskova E. --- разработка и реализация метода байесовской оптимизации для вывода демографической истории популяций по генетическим данным, проведение экспериментальных исследований (90\%); Borovitskiy~V. ---  рекомендации по постановке задач, выбору и обоснованию теоретических основ научного исследования (10\%).

    \item %\cite{zhernakova2020genome}: 
    В публикации~\cite{zhernakova2020genome} Noskova E. --- вывод демографической истории трех популяций современного человека (10\%); Ulyantsev~V. --- рекомендации по постановке задачи (5\%); остальные соавторы --- сбор и анализ генетических данных (85\%).

    \item %\cite{nikolic2022stepping}: 
    В публикации~\cite{nikolic2022stepping} Noskova E. --- вывод демографической истории двух и трех популяций голубых акул (10\%); остальные соавторы --- сбор и анализ генетических данных (90\%).

    \item %\cite{adrion2020community}: 
    В публикации~\cite{adrion2020community} Noskova E. --- разработка и тестирование программного обеспечения для симулирования генетических данных по демографической истории популяций (5\%); остальные соавторы --- разработка и тестирование программного обеспечения, проведение экспериментальных исследований (95\%).

    \item %\cite{lauterbur2023expanding}: 
    В публикации~\cite{lauterbur2023expanding} Noskova E. --- реализация демографических историй популяций в программном обеспечении для симулирования генетических данных по демографической истории популяций (5\%); остальные соавторы --- разработка программного обеспечения (95\%).
    
    \item %\cite{gower2022demes}: 
    В публикации~\cite{gower2022demes} Noskova E. --- разработка программного обеспечения для представления демографической истории популяций (5\%); остальные соавторы --- разработка программного обеспечения (95\%).\\
\end{enumerate}


\textbf{Структура диссертационной работы}

% Диссертация состоит из~введения, четырех глав, заключения и приложения.
% Полный объём диссертации составляет
% \formbytotal{TotPages}{страниц}{у}{ы}{}, включая
% \formbytotal{totalcount@figure}{рисун}{ок}{ка}{ков},
% \formbytotal{totalcount@table}{таблиц}{у}{ы}{} и восемь листингов.
Диссертация состоит из~введения, четырех глав, заключения и приложения.
Полный объём диссертации составляет
396 страниц, включая 120 рисунков,
16 таблиц и восемь листингов.
%\formbytotal{totalcount@algorithm}{листинг}{}{а}{ов}.
Список литературы содержит 171 наименований.\\