%!TEX root = ../dissertation.tex

\usepackage{xpatch}% or use http://tex.stackexchange.com/a/40705
\def\makenamesetup{%
  \def\bibnamedelima{~}%
  \def\bibnamedelimb{ }%
  \def\bibnamedelimc{ }%
  \def\bibnamedelimd{ }%
  \def\bibnamedelimi{ }%
  \def\bibinitperiod{.}%
  \def\bibinitdelim{~}%
  \def\bibinithyphendelim{.-}}    
\newcommand*{\makename}[3]{\begingroup\makenamesetup\xdef#1{#2, #3}\endgroup}

\newbibmacro*{name:bold}[2]{%
  \makename{\currname}{#1}{#2}%
  \makename{\findname}{\lastname}{\firstname}%
  \makename{\findinit}{\lastname}{\firstinit}%
  \ifboolexpr{ test {\ifdefequal{\currname}{\findname}}
            or test {\ifdefequal{\currname}{\findinit}} }{\bfseries}{}}

\newcommand*{\boldname}[3]{%
  \def\lastname{#1}%
  \def\firstname{#2}%
  \def\firstinit{#3}}
\boldname{}{}{}

\xpretobibmacro{name:family}{\begingroup\usebibmacro{name:bold}{#1}{#2}}{}{}
\xpretobibmacro{name:given-family}{\begingroup\usebibmacro{name:bold}{#1}{#2}}{}{}
\xpretobibmacro{name:family-given}{\begingroup\usebibmacro{name:bold}{#1}{#2}}{}{}
\xpretobibmacro{name:delim}{\begingroup\normalfont}{}{}

\xapptobibmacro{name:family}{\endgroup}{}{}
\xapptobibmacro{name:given-family}{\endgroup}{}{}
\xapptobibmacro{name:family-given}{\endgroup}{}{}
\xapptobibmacro{name:delim}{\endgroup}{}{}

%%% This is a place to put all definitions needed for this particular thesis to work

% Various definitions depending on how the bibliography is done

\addbibresource{dissertation.bib}
\DeclareSourcemap{		
    \maps{
        \map{
            \step[fieldsource=medium, match=\regexp{Электронный\s+ресурс}, final]
            \step[fieldset=media, fieldvalue=eresource]
        }
    }
}

% Definitions and includes related for the text

% \DeclareRobustCommand{\todo}{\textcolor{black}}
\newcommand{\revise}[1]{\textcolor{red}{#1}}
\graphicspath{{images/}}

\newcommand{\hamm}[1]{\mathcal{H}(#1)}
\newcommand{\tobinary}[1]{\mathcal{B}(#1)}
\newcommand{\theop}{\mathcal{X}}

\DeclareMathOperator*{\argmin}{arg\,min}
\DeclareMathOperator*{\argmax}{arg\,max}

\pgfplotscreateplotcyclelist{myplotcycle}{%
black\\%1
red!80!black\\%2
violet!80!black\\%3
gray!80!black\\%4
orange\\%5
brown!80!black\\%6
cyan!80!black\\%
green!70!black\\%7
green\\%8
blue!60!black\\%9
teal\\%10
magenta!70!black\\%11
yellow!90!black\\%12
}


\DeclarePairedDelimiter\abs{\lvert}{\rvert}
\DeclarePairedDelimiter\ceil{\lceil}{\rceil}
\DeclarePairedDelimiter\floor{\lfloor}{\rfloor}
\definecolor{darkred}{rgb}{0.7,0.1,0.1}
\definecolor{middarkgrey}{rgb}{0.35,0.35,0.35}
\definecolor{darkblue}{rgb}{0.1,0.1,0.5}

\usepackage[colorinlistoftodos,prependcaption,textsize=tiny]{todonotes}
\usepackage[normalem]{ulem}
\usepackage{url}

\usepackage{tikz}
\usetikzlibrary{shapes,arrows, automata, positioning, arrows}
\usepackage{pstricks}
% \tikzset{
%   ->, % makes the edges directed
%   >=stealth', % makes the arrow heads bold
%   node distance=1.9cm, % specifies the minimum distance between two nodes. Change if necessary.
%   every state/.style={thick, fill=gray!10, minimum size = 0pt}, % sets the properties for each ’state’ node
%   initial text=$ $, % sets the text that appears on the start arrow
% }

% \newcommand{\inote}[1]{\medskip\noindent$[$\textcolor{darkred}{Илья}: \emph{\textcolor{middarkgrey}{#1}}$]$\medskip}
% \newcommand{\inote}[1]{\todo[linecolor=Plum,backgroundcolor=Plum!25,bordercolor=Plum,inline]{#1}}
\newcommand{\needtodo}[1]{\todo[linecolor=Red,backgroundcolor=Red!25,bordercolor=Red,inline]{#1}}

\newcommand{\inote}[1]{}
% \newcommand{\needtodo}[1]{}

\newcommand{\pto}{\mathrel{\ooalign{\hfil$\mapstochar$\hfil\cr$\to$\cr}}}
\newcommand{\chresults}[1]{\section*{Выводы по главе~#1}
\addcontentsline{toc}{section}{Выводы по главе~#1}
}
% \newcommand{\chresults}[1]{}
\newcommand{\insection}[1]{В \textbf{разделе~#1}}
\newcommand{\insectionen}[1]{In the \textbf{section~#1}}


% my definitions
% local definitions
\usepackage{xspace} % prevents eating spaces
\newcommand{\dadi}[0]{$\partial$a$\partial$i\xspace}
\newcommand{\moments}[0]{\textit{moments}\xspace}
\newcommand{\momentsLD}[0]{\textit{momentsLD}\xspace}
\newcommand{\momi}[0]{\textit{momi2}\xspace}
\newcommand{\fastsimcoal}[0]{\textit{fastsimcoal2}\xspace}
\newcommand{\stdpopsim}[0]{\textit{stdpopsim}\xspace}
\newcommand{\demes}[0]{\textit{demes}\xspace}
\newcommand{\gadma}[0]{GADMA\xspace}

\renewcommand{\v}{\relax\ifmmode\expandafter\boldsymbol\else\expandafter\textv\fi} % vector
\newcommand{\m}{\expandafter\mathbf} % matrix


% scale math mode
\newcommand\scalemath[2]{\scalebox{#1}{\mbox{\ensuremath{\displaystyle #2}}}}

% tablenotes
\usepackage[flushleft]{threeparttable}

% start section after all figures and tables of previous section
\usepackage{placeins}  % \FloatBarrier

% P is centered column in table with sizes
\newcolumntype{P}[1]{>{\centering\arraybackslash}p{#1}}

% define new chi that is located inline
\DeclareRobustCommand{\rchi}{{\mathpalette\irchi\relax}}
\newcommand{\irchi}[2]{\raisebox{\depth}{$#1\chi$}} % inner command, used by \rchi